\documentclass[]{article}
\usepackage{lmodern}
\usepackage{amssymb,amsmath}
\usepackage{ifxetex,ifluatex}
\usepackage{fixltx2e} % provides \textsubscript
\ifnum 0\ifxetex 1\fi\ifluatex 1\fi=0 % if pdftex
  \usepackage[T1]{fontenc}
  \usepackage[utf8]{inputenc}
\else % if luatex or xelatex
  \ifxetex
    \usepackage{mathspec}
  \else
    \usepackage{fontspec}
  \fi
  \defaultfontfeatures{Ligatures=TeX,Scale=MatchLowercase}
\fi
% use upquote if available, for straight quotes in verbatim environments
\IfFileExists{upquote.sty}{\usepackage{upquote}}{}
% use microtype if available
\IfFileExists{microtype.sty}{%
\usepackage{microtype}
\UseMicrotypeSet[protrusion]{basicmath} % disable protrusion for tt fonts
}{}
\usepackage[margin=1in]{geometry}
\usepackage{hyperref}
\hypersetup{unicode=true,
            pdfborder={0 0 0},
            breaklinks=true}
\urlstyle{same}  % don't use monospace font for urls
\usepackage{longtable,booktabs}
\usepackage{graphicx,grffile}
\makeatletter
\def\maxwidth{\ifdim\Gin@nat@width>\linewidth\linewidth\else\Gin@nat@width\fi}
\def\maxheight{\ifdim\Gin@nat@height>\textheight\textheight\else\Gin@nat@height\fi}
\makeatother
% Scale images if necessary, so that they will not overflow the page
% margins by default, and it is still possible to overwrite the defaults
% using explicit options in \includegraphics[width, height, ...]{}
\setkeys{Gin}{width=\maxwidth,height=\maxheight,keepaspectratio}
\IfFileExists{parskip.sty}{%
\usepackage{parskip}
}{% else
\setlength{\parindent}{0pt}
\setlength{\parskip}{6pt plus 2pt minus 1pt}
}
\setlength{\emergencystretch}{3em}  % prevent overfull lines
\providecommand{\tightlist}{%
  \setlength{\itemsep}{0pt}\setlength{\parskip}{0pt}}
\setcounter{secnumdepth}{0}
% Redefines (sub)paragraphs to behave more like sections
\ifx\paragraph\undefined\else
\let\oldparagraph\paragraph
\renewcommand{\paragraph}[1]{\oldparagraph{#1}\mbox{}}
\fi
\ifx\subparagraph\undefined\else
\let\oldsubparagraph\subparagraph
\renewcommand{\subparagraph}[1]{\oldsubparagraph{#1}\mbox{}}
\fi

%%% Use protect on footnotes to avoid problems with footnotes in titles
\let\rmarkdownfootnote\footnote%
\def\footnote{\protect\rmarkdownfootnote}

%%% Change title format to be more compact
\usepackage{titling}

% Create subtitle command for use in maketitle
\newcommand{\subtitle}[1]{
  \posttitle{
    \begin{center}\large#1\end{center}
    }
}

\setlength{\droptitle}{-2em}

  \title{}
    \pretitle{\vspace{\droptitle}}
  \posttitle{}
    \author{}
    \preauthor{}\postauthor{}
    \date{}
    \predate{}\postdate{}
  

\begin{document}

\hypertarget{does-ethnic-density-influence-community-participation-in-local-running-events-a-case-of-parkrun}{%
\section{Does ethnic density influence community participation in local
running events?: a case of
parkrun}\label{does-ethnic-density-influence-community-participation-in-local-running-events-a-case-of-parkrun}}

\emph{Smith R\textsuperscript{1},Schneider PP\textsuperscript{1}, Bullas
A\textsuperscript{2}, Bayley T\textsuperscript{1}, Haake
SSJ\textsuperscript{2}, Brennan A\textsuperscript{1}, Goyder
E\textsuperscript{1}}

\begingroup\small

\emph{\textsuperscript{1}School of Health and Related Research,
University of Sheffield}\\
\emph{\textsuperscript{2}Sheffield Hallam University Advanced Wellbeing
Centre} \endgroup

\textbf{Keywords}: \texttt{parkrun}, \texttt{Participation},
\texttt{Physical\ Activity}, \texttt{Deprivation},
\texttt{Ethnic\ density}.

\textbf{Intended Journal}: Journal of Racial \& Ethnic Health
Disparities, Sports Medicine Open,Behavioural and Social Sciences,
Journal Of Epi \& Comm Health.

\begin{center}\rule{0.5\linewidth}{\linethickness}\end{center}

\hypertarget{thumbnail-sketch}{%
\subsubsection{Thumbnail Sketch}\label{thumbnail-sketch}}

\hypertarget{what-is-already-known-on-this-subject}{%
\paragraph{What is already known on this
subject?}\label{what-is-already-known-on-this-subject}}

parkrun organise weekly 5km running and walking events at parks across
the world. Recent research has shown that despite equitable geographical
access to parkrun events in England, participation is much lower in more
deprived areas.

\hypertarget{what-this-study-adds}{%
\paragraph{What this study adds?}\label{what-this-study-adds}}

This study uses regression modelling techniques to better understand the
relative influence of geographical access, deprivation and ethnic
density on parkrun participation rates in local communities. It finds
that areas with higher ethnic density tend to have lower participation
rates, even when controlling for deprivation.

\hypertarget{policy-implications}{%
\paragraph{Policy implications}\label{policy-implications}}

Identifying why particular communities are less likely to engage in
parkrun, and finding ways to improve participation from these
commmunities is likely to both improve overall population health and
reduce inequalities.

\begin{center}\rule{0.5\linewidth}{\linethickness}\end{center}

\hypertarget{abstract}{%
\section{Abstract}\label{abstract}}

parkrun has been successful in encouraging people in England to
participate in their weekly 5km running and walking events. However,
there is substantial heterogeneity in parkrun participation across
different communities in England: after controlling for travel
distances, deprived communities have significantly lower participation
rates (Schneider et al. 2019).

This paper aims to expand on previous findings by investigating ethnic
disparities in parkrun participation. We combined geo-spatial data
available through the ONS with participation data provided by parkrun,
and fitted multivariable Poisson regression models to study the effect
of ethnic density on participation rates at the Lower layer Super Output
Level.

We find that areas with higher ethnic density have lower participation
rates. This effect is independent of deprivation. An opportunity exists
for parkrun to engage with these communities and reduce potential
barriers to participation.

\hypertarget{introduction}{%
\section{Introduction}\label{introduction}}

parkrun is a collection of free mass participation 5km running events
that takes place every Saturday morning. There are currently over 500
locations in England, with a combined weekly attendance of over 100,000.
parkrun has been identified as being successful at engaging with
individuals who may not otherwise have taken part in organised physical
activity (Haake 2018; Stevinson and Hickson 2013), and there is some
evidence that it has increased overall physical activity levels in
participants (Stevinson and Hickson 2018). Overall, there is a consensus
that parkrun has huge public health potential (Reece et al. 2019).

However, qualitative research from Sheffield (Goyder et al. 2018) and
more broadly the United Kingdom (Fullagar et al. 2019) identified that
parkruns located in more deprived areas have lower attendances, and that
ethnic diversity in parkrun was limited. This leads to concern that as
with many public health interventions, parkrun is ``likely to be
responsible for significant intervention generated inequalities in
uptake of opportunities for physically active recreation'' (Goyder et
al. 2018).

Undertaking quantitative analysis of the determinants of participation
in parkrun is therefore long overdue. Aside from a single previous study
from Australia (Cleland et al. 2019), with substantial limitations
including, as noted by the authors, that ``The sample was limited to a
non-random sample of parkrun participants in one State of Australia and
may not be generalizable to other parkrun populations.'' (p.21), no
other studies have attempted to identify the determinants of
participation in parkrun.

Our previous work attempting to locate the optimal parks in which to
locate the next 200 parkrun events in England, revealed that there is
substantial heterogeneity in parkrun participation across different
communities in England: after controlling for geographical distance to
nearest event, deprived communities have significantly lower
participation rates (Schneider et al. 2019). This analysis was able to
quantify, for the first time, how participation in parkrun varied in
different communities in England. However, the analysis was interested
only in the relationship between participation, access and deprivation
and did not consider other potential determinants of participation in
parkrun. This paper extends this part of our previous analysis to better
understand how access, deprivation, ethnic density, population density
and age of the community influence parkrun participation in England.

Our initial hypothesis was that areas with a higher ethnic density, a
higher proportion of non-White-British residents, would have lower
participation in parkrun. We predicted this because 1) there is strong
evidence from survey data that non-White-British individuals in England
are less likely to be physcially active, and to engage in sport in
general (Rowe and Champion 2000), and 2) previous research has
identified low levels of ethnic minority participation in parkrun
(Goyder et al. 2018; Fullagar et al. 2019).

\hypertarget{methods}{%
\section{Methods}\label{methods}}

Data was obtained from multiple sources at the Lower layer Super Output
Level (LSOA). There are 32,844 LSOAs in England, each of which is a
geographical area containing around 1,000-3,000 people.

\href{https://www.parkrun.org.uk/}{parkrunUK} provided data on the
number of parkrun finishers from each LSOA in England between the 1st
January and 10th December 2018, we use the number of finishers as a
proxy for parkrun participation, althoough we appreciate that people
participate in parkrun in other ways (e.g.~volunteering). We also used
parkrun event location data publicly available on parkrun website
\href{https://www.parkrun.org.uk/}{here}.

The rest of the data, including Index of Multiple Deprivation (IMD)
Score, Ethnic Density, Rural-Urban Classification, Population Density,
Percent Working Age and LSOA centroids were obtained from the Office of
National Statistics (ONS). Full sources are listed in the table below,
and all ONS data is provided open source on the author's GitHub page.

\begin{longtable}[]{@{}lll@{}}
\toprule
\begin{minipage}[b]{0.22\columnwidth}\raggedright
Variable\strut
\end{minipage} & \begin{minipage}[b]{0.33\columnwidth}\raggedright
Description\strut
\end{minipage} & \begin{minipage}[b]{0.37\columnwidth}\raggedright
Source\strut
\end{minipage}\tabularnewline
\midrule
\endhead
\begin{minipage}[t]{0.22\columnwidth}\raggedright
run\_count\strut
\end{minipage} & \begin{minipage}[t]{0.33\columnwidth}\raggedright
number of runs from each LSOA in England between 1st January and 10th
December 2018\strut
\end{minipage} & \begin{minipage}[t]{0.37\columnwidth}\raggedright
parkrunUK\strut
\end{minipage}\tabularnewline
\begin{minipage}[t]{0.22\columnwidth}\raggedright
imd\strut
\end{minipage} & \begin{minipage}[t]{0.33\columnwidth}\raggedright
IMD scores for each LSOA\strut
\end{minipage} & \begin{minipage}[t]{0.37\columnwidth}\raggedright
\href{https://assets.publishing.service.gov.uk/government/uploads/system/uploads/attachment_data/file/835115/IoD2019_Statistical_Release.pdf}{ONS}\strut
\end{minipage}\tabularnewline
\begin{minipage}[t]{0.22\columnwidth}\raggedright
total\_pop\strut
\end{minipage} & \begin{minipage}[t]{0.33\columnwidth}\raggedright
total number of individuals in each LSOA\strut
\end{minipage} & \begin{minipage}[t]{0.37\columnwidth}\raggedright
\href{https://www.ons.gov.uk/peoplepopulationandcommunity/populationandmigration/populationestimates/datasets/lowersuperoutputareamidyearpopulationestimates}{ONS}.\strut
\end{minipage}\tabularnewline
\begin{minipage}[t]{0.22\columnwidth}\raggedright
pop\_density\strut
\end{minipage} & \begin{minipage}[t]{0.33\columnwidth}\raggedright
population density for each LSOA\strut
\end{minipage} & \begin{minipage}[t]{0.37\columnwidth}\raggedright
\href{https://www.ons.gov.uk/peoplepopulationandcommunity/populationandmigration/populationestimates/datasets/lowersuperoutputareapopulationdensity}{ONS}\strut
\end{minipage}\tabularnewline
\begin{minipage}[t]{0.22\columnwidth}\raggedright
rural\_urban\strut
\end{minipage} & \begin{minipage}[t]{0.33\columnwidth}\raggedright
Rural-Urban Classification\strut
\end{minipage} & \begin{minipage}[t]{0.37\columnwidth}\raggedright
\href{https://www.gov.uk/government/statistics/2011-rural-urban-classification-of-local-authority-and-other-higher-level-geographies-for-statistical-purposes}{ONS}\strut
\end{minipage}\tabularnewline
\begin{minipage}[t]{0.22\columnwidth}\raggedright
perc\_bme\strut
\end{minipage} & \begin{minipage}[t]{0.33\columnwidth}\raggedright
Ethnic Density: percent of population non-white-british\strut
\end{minipage} & \begin{minipage}[t]{0.37\columnwidth}\raggedright
\href{https://www.ons.gov.uk/peoplepopulationandcommunity/culturalidentity/ethnicity/datasets/2011censussmallpopulationtablesforenglandandwales}{ONS}\strut
\end{minipage}\tabularnewline
\begin{minipage}[t]{0.22\columnwidth}\raggedright
mn\_dstn\strut
\end{minipage} & \begin{minipage}[t]{0.33\columnwidth}\raggedright
distance from LSOA centroid to nearest parkrun\strut
\end{minipage} & \begin{minipage}[t]{0.37\columnwidth}\raggedright
derived from
\href{http://census.ukdataservice.ac.uk/get-data/boundary-data.aspx}{ONS}\strut
\end{minipage}\tabularnewline
\begin{minipage}[t]{0.22\columnwidth}\raggedright
perc\_non\_working\_age\strut
\end{minipage} & \begin{minipage}[t]{0.33\columnwidth}\raggedright
derived from ONS data on age-groups in each LSOA\strut
\end{minipage} & \begin{minipage}[t]{0.37\columnwidth}\raggedright
\href{https://www.ons.gov.uk/peoplepopulationandcommunity/populationandmigration/populationestimates/datasets/lowersuperoutputareamidyearpopulationestimates}{ONS}\strut
\end{minipage}\tabularnewline
\begin{minipage}[t]{0.22\columnwidth}\raggedright
run\_rate\strut
\end{minipage} & \begin{minipage}[t]{0.33\columnwidth}\raggedright
derived from run\_count and LSAO populations\strut
\end{minipage} & \begin{minipage}[t]{0.37\columnwidth}\raggedright
derived\strut
\end{minipage}\tabularnewline
\bottomrule
\end{longtable}

After merging these datasets we had detailed data on 32,844 LSOAs,
including participation and several characteristics of the LSOAs which
we hypothesised may influence participation. Since previous work has
found corelations between participation and deprivation, distance to
nearest event, and population density we included all of these
variables. We also extended the analysis to include ethnic density (we
use the percent of the population that reported being non-White-British
as a proxy for ethnic density) and the percent of the population of
working age. We are interested in ethnic density as we hypothesised that
areas with higher ethnic density would have lower participation rates,
all else being equal. We included the percent of the population that is
working age as a control to limit for the effect of populations heavily
skewed toward older people (e.g.~care homes), or very young people (e.g
orphanages/immediately around special needs schools)). Since
participation in parkrunUK is dominated by those aged 20-60 (Haake 2018)
we felt this was justified.

We used R Software Environment (R Core Team 2018) to produce simple
summary statistics and viewed simple correlations between the variables
described using packages created by Wei and Simko (2017). We then moved
on to estimate the partial correlation coefficients between the
variables, again with Wei and Simko (2017), before running a poisson
regression model on scaled data to estimate the relative influence of
several determinants on parkrun participation. All graphs are created
with ggplot2 (Wickham 2016), the results of the poisson regression model
were displayed using the stargazer package (Hlavac 2018).

\hypertarget{results}{%
\section{Results}\label{results}}

\hypertarget{descriptive-statistics}{%
\subsection{Descriptive Statistics}\label{descriptive-statistics}}

Participation in parkrun varies between LSOAs. Around half of all
communities (LSOA) average less than 1 finisher per week per 1000
people. Approximately a quarter average between 1 and 2 runs, and around
an eighth between 2 and 3 runs. There is considerable variation in
ethnic density, with most LSOA having a large majority of White-British
residents, and few areas having over 50\% non-White-British residents.
Deprivation is positively skewed, meaning that most areas are not
deprived, with a few very deprived areas. Finally, around 70\% of LSOAs
are within 5km, a parkrun, of a parkrun. Again this is positively skewed
with most LSAO being within 3-4km.

\includegraphics{What_determines_LSOA_parkrun_participation_files/figure-latex/unnamed-chunk-1-1.pdf}

\newpage

\hypertarget{correlation-matrix}{%
\subsection{Correlation Matrix}\label{correlation-matrix}}

There is a negative correlation between particpation (run\_count) and:
deprivation (imd), distance to nearest parkrun (mn\_distance),
population density (pop\_density) and ethnic density (perc\_bme). Ethnic
density is strongly postively correlated with population density,
negatively correlated with percent non-working age, and moderately
positvely correlated with IMD suggesting that areas with higher ethnic
density are more densely populated overall, more deprived and have fewer
older people.

\includegraphics{What_determines_LSOA_parkrun_participation_files/figure-latex/unnamed-chunk-2-1.pdf}
\newpage

\newpage

\hypertarget{colour-plot-for-ruralurban.}{%
\subsection{Colour plot for
rural/urban.}\label{colour-plot-for-ruralurban.}}

We are interested in particular in how participation in parkrun varies
by ethnic density and IMD of the community. The colour plots below show
the participation levels for LSOA by deprivation and ethnic density for
Urban Major, Urban Minor and Rural areas. The yellow areas are those
with high rates of participation, and the dark blue are areas with low
levels of participation.

The plot shows that participation is generally greatest in areas that
have low levels of deprivation and low levels of ethnic density (bottom
left), and lowest in areas with high levels of deprivation and high
ethnic density (top-right). Areas with either high deprivation, or high
ethnic density, tended to have low participation, suggesting that both
are important independently. The relationship was robust to Urban Major
areas and Urban Minor areas but did not hold in Rural areas where data
was more limited. It is important to note that we do not control for
other factors, such as the age of residents or the population density
and there are therefore many confounding factors.

\includegraphics{What_determines_LSOA_parkrun_participation_files/figure-latex/unnamed-chunk-4-1.pdf}

\newpage

\hypertarget{partial-correlation-matrix}{%
\subsection{Partial correlation
matrix}\label{partial-correlation-matrix}}

In order to control for confounders, we created a partial correlation
matrix. This estimates the correlation between the variables in the
analysis holding all other variables constant.

Looking at the top row, it is clear that participation rates and
deprivation is strongly correlated, but less so now we are controlling
for other variables. Distance to nearest parkrun event is also strongly
correlated, as was the case in our previous analysis (Schneider et al.
2019). Ethnic density is also negatively correlated, suggesting that
some of the relationship between participation and IMD is attributable
to the ethnic density of the area.

\emph{This no longer holds for some reason}

\includegraphics{What_determines_LSOA_parkrun_participation_files/figure-latex/unnamed-chunk-6-1.pdf}

\newpage

\hypertarget{poisson-model}{%
\subsection{Poisson Model}\label{poisson-model}}

Poisson models are commonly used to estimate count data, where values
are constrained to be above 0. In this case, because it is not possible
for a community to have negative participation we use a Poisson
regression.

The results of the poisson regression are not notoriously easy to
interpret, but show that areas with a higher ethnic density have lower
participation rates, even when controlling for the effect of deprivation
and distance to events. The effect is smaller than deprivation and
distance, but still material and significant.

\begin{table}[!htbp] \centering 
  \caption{Regression Results} 
  \label{} 
\begin{tabular}{@{\extracolsep{5pt}}lccc} 
\\[-1.8ex]\hline 
\hline \\[-1.8ex] 
 & \multicolumn{3}{c}{\textit{Dependent variable:}} \\ 
\cline{2-4} 
\\[-1.8ex] & \multicolumn{3}{c}{run\_count} \\ 
 & Original scale & Scaled - min model & Scaled - full model \\ 
\\[-1.8ex] & (1) & (2) & (3)\\ 
\hline \\[-1.8ex] 
 imd & $-$0.034$^{***}$ & $-$0.532$^{***}$ & $-$0.519$^{***}$ \\ 
  & ($-$0.034, $-$0.034) & ($-$0.533, $-$0.530) & ($-$0.520, $-$0.517) \\ 
  & & & \\ 
 pop\_density & $-$0.070$^{***}$ &  & $-$0.106$^{***}$ \\ 
  & ($-$0.071, $-$0.069) &  & ($-$0.108, $-$0.105) \\ 
  & & & \\ 
 mn\_dstn & $-$0.112$^{***}$ & $-$0.424$^{***}$ & $-$0.475$^{***}$ \\ 
  & ($-$0.112, $-$0.111) & ($-$0.426, $-$0.423) & ($-$0.477, $-$0.474) \\ 
  & & & \\ 
 perc\_non\_working\_age & $-$0.134$^{***}$ &  & $-$0.011$^{***}$ \\ 
  & ($-$0.148, $-$0.121) &  & ($-$0.012, $-$0.009) \\ 
  & & & \\ 
 perc\_bme & $-$1.524$^{***}$ & $-$0.320$^{***}$ & $-$0.285$^{***}$ \\ 
  & ($-$1.532, $-$1.515) & ($-$0.322, $-$0.319) & ($-$0.287, $-$0.284) \\ 
  & & & \\ 
 Constant & $-$0.737$^{***}$ & $-$2.798$^{***}$ & $-$2.804$^{***}$ \\ 
  & ($-$0.747, $-$0.727) & ($-$2.800, $-$2.797) & ($-$2.805, $-$2.803) \\ 
  & & & \\ 
\hline \\[-1.8ex] 
Observations & 32,844 & 32,844 & 32,844 \\ 
Log Likelihood & $-$1,231,308.000 & $-$1,245,048.000 & $-$1,231,308.000 \\ 
Akaike Inf. Crit. & 2,462,628.000 & 2,490,104.000 & 2,462,628.000 \\ 
\hline 
\hline \\[-1.8ex] 
\textit{Note:}  & \multicolumn{3}{r}{$^{*}$p$<$0.1; $^{**}$p$<$0.05; $^{***}$p$<$0.01} \\ 
\end{tabular} 
\end{table}

\newpage

\hypertarget{discussion}{%
\section{Discussion}\label{discussion}}

Previous analysis by
\href{https://www.medrxiv.org/content/early/2019/08/29/19004143.full.pdf}{Scheider
et al., 2019} has showed that participation in parkrun is lower in more
deprived communities, and communities that are further from their
nearest parkrun. This paper extends that analysis to include other
community level characteristics which were hypothesized to influence
parkrun participation. Our findings show that some of the negative
effect on participation previously attributed to deprivation can
actually be attributed to differences in participation by ethnic
minotiries.

We find that communities with higher ethnic density were less likely to
participate in parkrun, even when controlling for deprivation. Since
ethnic density and deprivation are positively correlated, it is likely
that some of the effect attributed to deprivation in our previous
analysis
\href{https://www.medrxiv.org/content/early/2019/08/29/19004143.full.pdf}{Scheider
et al., 2019} is is attributable to ethnicity.

parkrun's mission states that they aim to increase levels of physical
activity in deprived communities. Our findings indicate that
participation in deprived communities with ethnic density is
particularly low. Further research could be undertaken to ascertain
trends in participation from different groups in society, allowing
parkrun to monitor the effectiveness of their efforts to reach minority
communities. More research is needed to understand the barriers to
attending parkrun for members of those communities.

\hypertarget{limitations}{%
\section{Limitations}\label{limitations}}

This analysis is ecological and therefore it is not possible to make
conclusions at an individual level without making an ecological
inference fallacy. We have been careful throught to make conclusions at
the level of the LSOA, rather than te individual. Nevertheless, given
that the evidence at the individual level points to lower participation
in organised sport by those from ethnic minority backgrounds (insert
REF), we think it is likely that the same effect exists at the
individual level.

Our dependent variable is the number of runs by residents of each LSOA.
This is a count variable where each walk/run finish is treated equally
(e.g.~10 runs by one person is equal to 1 run by 10 people). We cannot
draw inferences on the number of people who took part within each LSOA
at some point in the year, but instead focus on the total finisher
count.

We controlled for several variables which we thought would influence
participation, it is possible that there are other confounding factors
which have not been included.

\hypertarget{conclusion}{%
\section{Conclusion}\label{conclusion}}

Community level participation in parkrun is negatively correlated with
deprivation, distance to nearest event and ethnic density. parkrun is
already in the process of increasing the number of events in England.
Understanding the other determinants of participation rates at the
community level, and identifying interventions to increase participation
is the obvious next step. Breaking down barriers to engagement in
parkrun has the potential to improve overall population physical activty
and therefore improve overall health and reduce health inequalities.

\hypertarget{references}{%
\section*{References}\label{references}}
\addcontentsline{toc}{section}{References}

\hypertarget{refs}{}
\leavevmode\hypertarget{ref-cleland2019exploring}{}%
Cleland, Verity, Meredith Nash, Melanie J Sharman, and Suzi Claflin.
2019. ``Exploring the Health-Promoting Potential of the `Parkrun'
Phenomenon: What Factors Are Associated with Higher Levels of
Participation?'' \emph{American Journal of Health Promotion} 33 (1):
13--23.

\leavevmode\hypertarget{ref-fullagar2019action}{}%
Fullagar, Simone, Sandra Petris, Julia Sargent, Stephanie Allen, M
Akhtar, and Gozde Ozakinci. 2019. ``Action Research with Parkrun Uk
Volunteer Organisers to Develop Inclusive Strategies.'' \emph{Health
Promotion International}, In--Press.

\leavevmode\hypertarget{ref-goyder2018p2}{}%
Goyder, EC, C Edmonds, A Sabey, D Lawrence, A Bullas, M Taylor, and S
Potter. 2018. ``P2 What Factors Predict Participation in a Mass
Community Physical Activity Programme? The Case of the Five Sheffield
`Parkruns'.'' BMJ Publishing Group Ltd.

\leavevmode\hypertarget{ref-haake2018parkrun}{}%
Haake, Steve. 2018. ``Parkrun: A New Model of Physical Activity for
Large Populations?'' \emph{The Sport and Exercise Scientist} 57: 18--19.

\leavevmode\hypertarget{ref-stargazer}{}%
Hlavac, Marek. 2018. \emph{Stargazer: Well-Formatted Regression and
Summary Statistics Tables}. Bratislava, Slovakia: Central European
Labour Studies Institute (CELSI).
\url{https://CRAN.R-project.org/package=stargazer}.

\leavevmode\hypertarget{ref-base_r}{}%
R Core Team. 2018. \emph{R: A Language and Environment for Statistical
Computing}. Vienna, Austria: R Foundation for Statistical Computing.
\url{https://www.R-project.org/}.

\leavevmode\hypertarget{ref-reece2019bright}{}%
Reece, Lindsey J, Helen Quirk, Chrissie Wellington, Steve J Haake, and
Fiona Wilson. 2019. ``Bright Spots, Physical Activity Investments That
Work: Parkrun; a Global Initiative Striving for Healthier and Happier
Communities.'' \emph{Br J Sports Med} 53 (6): 326--27.

\leavevmode\hypertarget{ref-rowe2000sport}{}%
Rowe, Nick, and Ross Champion. 2000. ``Sport Participation and Ethnicity
in England.'' \emph{Headline Findings. Sport England, London}.

\leavevmode\hypertarget{ref-schneider2019}{}%
Schneider, Paul P, Robert A Smith, Alice M Bullas, Thomas Bayley, Steve
SJ Haake, Alan Brennan, and Elizabeth Goyder. 2019. ``Where Should New
Parkrun Events Be Located? Modelling the Potential Impact of 200 New
Events on Socio-Economic Inequalities in Access and Participation.''
\emph{medRxiv}, 19004143.

\leavevmode\hypertarget{ref-stevinson2013exploring}{}%
Stevinson, Clare, and Mary Hickson. 2013. ``Exploring the Public Health
Potential of a Mass Community Participation Event.'' \emph{Journal of
Public Health} 36 (2): 268--74.

\leavevmode\hypertarget{ref-stevinson2018changes}{}%
---------. 2018. ``Changes in Physical Activity, Weight and Wellbeing
Outcomes Among Attendees of a Weekly Mass Participation Event: A
Prospective 12-Month Study.'' \emph{Journal of Public Health}.

\leavevmode\hypertarget{ref-corrplot2017}{}%
Wei, Taiyun, and Viliam Simko. 2017. \emph{R Package "Corrplot":
Visualization of a Correlation Matrix}.
\url{https://github.com/taiyun/corrplot}.

\leavevmode\hypertarget{ref-ggplot2}{}%
Wickham, Hadley. 2016. \emph{Ggplot2: Elegant Graphics for Data
Analysis}. Springer-Verlag New York.
\url{https://ggplot2.tidyverse.org}.


\end{document}
